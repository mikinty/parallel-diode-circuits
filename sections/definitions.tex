\section{Definitions}

\begin{definition}
  An \textbf{ideal diode} is a circuit element that has infinite current when
  on, and 0 current when off. The threshold voltage\footnote{we sometimes call this the ``turn-on voltage''} for when it is on and off
  is called $\VD$ and is measured from the $+$ to the $-$ terminal.

  \begin{figure}[H]
    \centering
    \begin{circuitikz}
      \draw (0, 0) node[above] {$+$} to[D*]
      (2, 0) node[above] {$-$};
    \end{circuitikz}
    \caption{A diode, with the positive and negative terminals marked.}
  \end{figure}
\end{definition}

\begin{property}
  For a chain of diodes $D_1, D_2, \dots, D_n$ in series, the turn on voltage for the
  chain is
  \begin{equation}
    V_D = \sum_{i=1}^n V_{D_i}
  \end{equation}
  \begin{figure}[H]
    \centering
    \begin{circuitikz}
      \draw (0, 0)  to[D*, l=$D_1$]
      (2, 0) to[D*, l=$D_2$]
      (4, 0)
      (4.5, 0) node[] {$\cdots$}
      (5, 0) to[D*, l=$D_n$] (7,0)
      ;

      \draw[decoration={brace,mirror},decorate]
      (7,1) -- node[above=6pt] {$\sum V_{D_i}$} (0,1)
      ;
    \end{circuitikz}
    \caption{Diode chain}
  \end{figure}
  \label{property:diode_series}
\end{property}

\begin{definition}
  A \textbf{voltage source} is a circuit element that maintains a voltage of
  $V$ between 2 nodes.

  In this paper, we will be using the following element to represent a voltage source.
  \begin{figure}[H]
    \centering
    \begin{circuitikz}
      \draw (0, 0) to[american voltage source, l=$\SI{5}{\volt}$] (2, 0);
    \end{circuitikz}
    \caption{A voltage source of $\SI{5}{\volt}$}
  \end{figure}
\end{definition}

\begin{definition}
  An \textbf{ammeter} is a device used to measure the current at some node of a circuit.
  We will use the notation
  \begin{equation}
    I(A_i)
  \end{equation}
  to describe the current measured by ammeter $A_i$.

  \begin{figure}[H]
    \centering
    \begin{circuitikz}
      \draw (0, 0) to[ammeter] (2, 0);
    \end{circuitikz}
    \caption{An ammeter}
  \end{figure}
\end{definition}

\begin{definition}
  A \textbf{switch} is a device that can be set to 2 configurations, either
  closed, which behaves as a wire in a circuit, or open, which will behave as
  an open circuit, so no current can flow through.
  \begin{figure}[H]
    \centering
    \begin{circuitikz}
      \draw (0, 0) to[nos] (2, 0);
      \draw (3, 0) to[ncs] (5, 0);
    \end{circuitikz}
    \caption{An open switch (left) and a closed switch (right)}
  \end{figure}
\end{definition}