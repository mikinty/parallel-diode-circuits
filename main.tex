\documentclass{article}

%%% PREAMBLE
\usepackage{arxiv}
\usepackage{util}

\usepackage[utf8]{inputenc} % allow utf-8 input
\usepackage[T1]{fontenc}    % use 8-bit T1 fonts
\usepackage{hyperref}       % hyperlinks
\usepackage{url}            % simple URL typesetting
\usepackage{booktabs}       % professional-quality tables
\usepackage{amsfonts}       % blackboard math symbols
\usepackage{nicefrac}       % compact symbols for 1/2, etc.
\usepackage{microtype}      % microtypography
\usepackage{lipsum}		% Can be removed after putting your text content
\usepackage{graphicx}
\usepackage{natbib}
\usepackage{doi}



\title{Asymptotically Faster Circuit Topologies}

%\date{September 9, 1985}	% Here you can change the date presented in the paper title
%\date{} 					% Or removing it

\author{
  Michael~You\thanks{Github: mikinty} \\
  \texttt{youmichaelc@gmail.com}
}

% Uncomment to override  the `A preprint' in the header
\renewcommand{\headeright}{Michael You}
\renewcommand{\undertitle}{}
\renewcommand{\shorttitle}{Fast Circuit Topologies}

%%% Add PDF metadata to help others organize their library
%%% Once the PDF is generated, you can check the metadata with
%%% $ pdfinfo template.pdf
\hypersetup{
pdftitle={A template for the arxiv style},
pdfsubject={q-bio.NC, q-bio.QM},
pdfauthor={Michael~You},
pdfkeywords={Keywords},
}

\begin{document}
\maketitle

\begin{abstract}
	We are familiar with using the Turing paradigm to create computers,
	representing information in binary bits. However, there are more natural
	ways to solve problems without using bits. For example, if you have a
	graph, why not just find a natural way to represent it so the computation
	can be done better? That's the motivation of the paper. We will explore
	designing more natural circuit topologies to solve problems, and analyze
	their computational complexity. We will find that for sorting, we can
	come up with an algorithm that is faster than existing algorithms, and
	shortest paths in graphs that we can perform asymptotically better than
	existing algorithms. We will then explore the possibility of extending more
	natural circuit topologies to solve other problems faster, and applications
	of using this circuit topology technique to solve algorithmic problems in general.
\end{abstract}


% keywords can be removed
\keywords{Computational Complexity \and Circuits \and Diodes \and Graphs}


\section{Introduction}

Turing machines

efficient for information storage, but is it the best way to compute things? Best way to represent everything?

There can be more natural ways to represent problems.

\section{Definitions}

\begin{definition}
	An \textbf{ideal diode} is a circuit element that has infinite current when
	on, and 0 current when off. The threshold voltage\footnote{we sometimes call this the ``turn-on voltage''} for when it is on and off
	is called $\VD$ and is measured from the $+$ to the $-$ terminal.

	\begin{figure}[H]
		\centering
		\begin{circuitikz}
			\draw (0, 0) node[above] {$+$} to[D*]
			(2, 0) node[above] {$-$};
		\end{circuitikz}
		\caption{A diode, with the positive and negative terminals marked.}
	\end{figure}
\end{definition}

\begin{property}
	For a chain of diodes $D_1, D_2, \dots, D_n$ in series, the turn on voltage for the
	chain is
	\begin{equation}
		V_D = \sum_{i=1}^n V_{D_i}
	\end{equation}
	\begin{figure}[H]
		\centering
		\begin{circuitikz}
			\draw (0, 0)  to[D*, l=$D_1$]
				(2, 0) to[D*, l=$D_2$]
				(4, 0)
				(4.5, 0) node[] {$\cdots$}
				(5, 0) to[D*, l=$D_n$] (7,0)
			;

			\draw[decoration={brace,mirror},decorate]
				(7,1) -- node[above=6pt] {$\sum V_{D_i}$} (0,1)
			;
		\end{circuitikz}
		\caption{Diode chain}
	\end{figure}
	\label{property:diode_series}
\end{property}

\begin{definition}
	A \textbf{voltage source} is a circuit element that maintains a voltage of
	$V$ between 2 nodes.

	In this paper, we will be using the following element to represent a voltage source.
	\begin{figure}[H]
		\centering
		\begin{circuitikz}
			\draw (0, 0) to[american voltage source, l=$\SI{5}{\volt}$] (2, 0);
		\end{circuitikz}
		\caption{A voltage source}
	\end{figure}
\end{definition}

\begin{definition}
	An \textbf{ammeter} is a device used to measure the current at some node of a circuit.
	We will use the notation
	\begin{equation}
		I(A_i)
	\end{equation}
	to describe the current measured by ammeter $A_i$.

	\begin{figure}[H]
		\centering
		\begin{circuitikz}
			\draw (0, 0) to[ammeter] (2, 0);
		\end{circuitikz}
		\caption{An ammeter}
	\end{figure}
\end{definition}

\begin{definition}
	A \textbf{switch} is a device that can be set to 2 configurations, either
	closed, which behaves as a wire in a circuit, or open, which will behave as
	an open circuit, so no current can flow through.
	\begin{figure}[H]
		\centering
		\begin{circuitikz}
			\draw (0, 0) to[nos] (2, 0);
		\end{circuitikz}
		\caption{An open switch}
	\end{figure}
\end{definition}

\section{Circuit Topologies}
Here, we will explore two circuit topologies, their computational complexity in the problem they solve, and their costs.
\begin{itemize}
	\item Sorting numbers
	\item Shortest path in a graph
\end{itemize}


\subsection{Sorting}
We define the sorting problem to be

Given a list of $n$ numbers,

The fastest algorithms for sorting numbers are

\begin{itemize}
	\item \textbf{QuickSort:} $O(n \log n)$
	\item \textbf{Radix Sort:} $O(b(n+k))$
\end{itemize}

\subsubsection{Circuit}
\begin{figure}[H]
	\centering
	\def\labelIndex{{1, 2, 3, n, 5}}
	\def\yValues{{5, 3, 1, -3, -5}}
	\begin{circuitikz}
		\draw (0, 0) to[american voltage source, l_=$V$]
			(0, -2) node[sground] {};
		\draw (0, 0) to[short] (2, 0)
			(2, 5) to[short] (2, -5)
			(6.5, 5) to[short] (6.5, -5)
			(6.5, 0) to[short] (7, 0) node[sground, rotate=90] {}
		;

		% Diodes
		\foreach \y/\i in {5/1, 3/2, 1/3, -3/n-1, -5/n}
			\draw (2, \y) to[nos, l=$S_{\i}$]
				(3.5, \y) to[ammeter, l=$A_{\i}$]
				(5, \y) to[D*, l=$D_{\i}$] (6.5, \y)
			;

		\draw (4.25, -1) node[] {$\cdots$};
	\end{circuitikz}
	\caption{Circuit for sorting numbers}
	\label{fig:circuit_sorting}
\end{figure}


\subsubsection{Algorithm}

\begin{lstlisting}[label={alg:sort}]
	SORT:
		// Build circuit
		Get a voltage source $V$, with nodes $A$ at the positive, $B$ at the negative
		for $i = 1:n$:
			Build a series circuit of a switch $S_i$, ammeter $A_i$, and diode $D_i$
			and put it between $A, B$

		// Sort
		queue Q = []
		set $V = \Vmax$
		while $\exists S_i$ that is on:
			for $i$ such that $I(A_i) > 0$:
				turn off $S_i$
				Q.push(V_i)

		return Q
\end{lstlisting}

We will now prove this algorithm is correct and runs in $O(n)$ time.

First, we will prove the following lemma,

\begin{lemma}
	If there are diodes $D_1, D_2, \dots, D_n$ in parallel, and a voltage of $V
	\geq \max(V_{D_1}, V_{D_2}, \dots, V_{D_n})$ is applied across these diodes,
	the voltage across the diodes is $\Vmin = \min(V_{D_1}, V_{D_2}, \dots, V_{D_n})$,
	and only diodes $D_i$ such that $V_{D_i} = \Vmin$ are on.
\end{lemma}

\begin{proof}
	We will prove this statement by contradiction.

	Suppose some other $D_k$ where $V_{D_k} > \Vmin$ is on. Then the voltage
	across the diodes is $V_{D_k}$. But since we know that $V_{D_k} >
	\Vmin$, it must be the case that some $D_i$ with $V_{D_i} = \Vmin$ is also
	on. However, if $D_i$ is on, then the voltage across the diodes is $V_{D_i}
	< V_{D_k}$, and therefore $D_k$ cannot be on. We have reached a
	contradiction, and therefore only diodes $D_I$ with $V_{D_i} = \Vmin$ can
	be on.
	\label{lemma:smallest_diode}
\end{proof}

Now, the proof for the sorting algorithm is simple,

\begin{theorem}
	Algorithm \ref{alg:sort} correctly sorts diodes $D_1, D_2, \dots, D_n$ from
	smallest to greatest threshold voltage $V_{D_i}$.
\end{theorem}
\begin{proof}
	Once we have built our circuit, since $V = \Vmax$, we know at least one
	diode is turned on, as long as $\exists S_i$ that is on.
	For the diodes that are on, or have $I(A_i) > 0$, they will only be the
	diodes that are on, and these diodes, by Lemma \ref{lemma:smallest_diode},
	are the ones with the smallest threshold voltage in the set of diodes.

	Now, we can show this sorting algorithm works.

	\begin{itemize}
		\item Initially, the queue $Q$ is empty, so it is trivially sorted
		\item On an arbitrary iteration with $Q$ sorted, we only add diodes
		with $V_{D_i}$ the smallest in the set of remaining diodes with their switch on.
		When we add these diodes to the queue, $V_{D_i}$ must be larger than the
		previous diode $V_{D_j}$ added, since when $D_j$ was added, $D_i$ was not on.
		Therefore, $Q$ remains sorted in each iteration.
		\item On the last iteration, we are left with diodes that have a larger threshold voltage than all diodes currently in $Q$,
		so adding these last diodes to $Q$ still keeps $Q$ sorted.
	\end{itemize}
\end{proof}

\subsubsection{Complexity}

\begin{theorem}
	Algorithm \ref{alg:sort} runs in $O(n)$ time.
\end{theorem}

\begin{proof}
	Building the circuit takes 1 operation for putting the voltage source, and $3n$ operations for
	attaching the switch, ammeter, and diode in parallel.

	Then, to run the algorithm, we are just turning off $S_i$ one by one, $n$
	times, and each time adding $V_{D_i}$ to our queue, which is a total of 2
	operations.

	Therefore, the total number of operations is
	\begin{equation}
		1 + 3n + 2n = 5n+1 \in O(n).
	\end{equation}
\end{proof}

Notice that although the complexity is $O(n)$, which matches radix sort,
the constant $kn \in O(n)$ is for a $k \approx 2$, which outperforms radix sort.

\subsection{Shortest Path}
A natural extension to the sorting circuit is to create graphs out of diodes,
and then look for shortest paths.

\subsubsection{Circuit}

\subsubsection{Algorithm}

Before we prove the correctness of our algorithm,
we will show the following to make the proof easier to understand.

\begin{lemma}
\end{lemma}

\begin{proof}

\end{proof}

\section{Real-Life Implementations of Circuits}

To demonstrate that these ideas can work in practice, I built these circuits
in real life and tested them out.

\section{Discussion}

How do we get diodes of arbitrary threshold voltage?
This is unlikely, and can vastly limit the scopes of the problems we'd like to solve.

Physical limitations with electrons traveling through a wire.

What other problems can we solve with this idea?


\bibliographystyle{unsrtnat}
\bibliography{references}

\end{document}
